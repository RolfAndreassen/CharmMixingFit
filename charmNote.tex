\documentclass[12pt,pdflatex]{article}
\usepackage{color}
\usepackage[usenames,dvipsnames]{xcolor}
\usepackage{colortbl} 
\usepackage{graphicx}
\usepackage{colordvi}
\usepackage{amssymb}
\usepackage{ulem} 
\pagestyle{empty}
%\textheight 510pt
%\textwidth 683.4pt
%\oddsidemargin 0pt
%\evensidemargin 0pt
%\topmargin -82pt
\fboxrule0.05in
\fboxsep0.1in
\usepackage{fancyhdr}
\pagestyle{fancy}
\addtolength{\headheight}{2.5pt}
\lfoot{\thepage\ of \pageref{lastpage}}
%\cfoot{Rolf Andreassen}
\rfoot{University of Cincinnati} 

\newcommand{\mb}{\mathbf}
\boldmath 

\begin{document}
\begin{center}
{\LARGE Combining mixing results}
\end{center}

\section{Introduction}

The new $D^0\to K\pi$ result from LHCb provides a credibly powerful constraint
on mixing parameters. This note describes a fit in the style of HFAG to combine 
our result with previous measurements. 

\section{Chi-square calculation}

The purpose of our fit is to combine the errors on several different
measurements of the same parameters, where each measurement may have a
different relation to the underlying true mixing parameters (eg measuring
$(x'^2, y')$ in place of $(x, y)$), and where the numbers in each measurement
may be strongly correlated. To do so we construct an overall $\chi^2$ for
all the results:
\begin{eqnarray}
\chi^2 &=& \vec\epsilon^T \sigma^-1 \vec\epsilon
\end{eqnarray}
where the elements of $\vec\epsilon$ are given by
$\epsilon_i = m_i - p_i$. Here $\vec m$ is the list of measured
values from experiments, and $\vec p$ is a set of ``proposed'' values
for the mixing parameters; we use MINUIT to vary $\vec p$ so as
to minimise $\chi^2$. Finally, $\sigma$ is an $N\times N$ matrix where
$N$ is the number of measurements, with $\sigma_{ij} = e_i c_{ij} e_j$.
Here $e_i$ is the reported error on measurement $i$, and $c_{ij}$ is the
correlation coefficient between measurements $i$ and $j$. 

Notice that, if
the measurements are uncorrelated, then $\sigma$ reduces to
a diagonal matrix where the elements are the squares of the measurement
errors. In this case $\chi^2$ is simply the sum $\sum\limits_{i}\epsilon_i^2/e_i^2$,
that is, each element is the difference between a measurement
and the corresponding prediction, divided by the error on the measurement,
squared. In other words, if there are no correlations we recover
the usual chi-square goodness-of-fit metric. 

\section{Fit variants}

In full generality, we wish to fit for no less than seven underlying
related mixing parameters:
\begin{itemize}
\item $x$ and $y$, the normalised mass and width differences
\item $R_D^+$ and $R_D^-$, the ratios of rates
\item $\delta$, the strong phase difference
\item $|q/p$ and $\phi$, the magnitude and phase of the indirect CP violation. 
\end{itemize}
The observed inputs, however, are not all direct measurements of these
quantities. From $D^0\to K_S\pi\pi$ we get direct measurements of $x$, $y$, $|q/p|$ and $\phi$;
$D^0\to K\pi$ results also yield $R_D^\pm$ directly, although sometimes quoted
as $R_D &=& \frac{1}{2}(R_D^+ + R_D^-)$ and $A_D &=& \frac{R_D^+ - R_D^-}{R_D^+ + R_D^-}$.
However, we also measure the derived parameters $x'^{2(\pm)}$, $y'^{(\pm)}$, $y_{CP}$, and $A_\Gamma$,
defined as:
\begin{eqnarray}
x' &=& x\cos\delta + y\sin\delta \\
y' &=& y\cos\delta - x\sin\delta \\
x'^{\pm} &=& \left(\frac{1\pm A_M}{1\mp A_M}\right)^{1/4}\left(x'\cos\phi \pm y'\sin\phi\right) \\
y'^{\pm} &=& \left(\frac{1\pm A_M}{1\mp A_M}\right)^{1/4}\left(y'\cos\phi \mp x'\sin\phi\right) \\
2y_{CP}  &=& \left(|q/p|+|p/q|\right)y\cos\phi - \left(|q/p|-|p/q|\right)x\sin\phi\\
2A_\Gamma  &=& \left(|q/p|-|p/q|\right)y\cos\phi - \left(|q/p|+|p/q|\right)x\sin\phi\\
\end{eqnarray}
where the helper quantity $A_M$ is given by
\begin{eqnarray}
A_M &=& \frac{|q/p|^2 - |p/q|^2}{|q/p|^2 + |p/q|^2}. 
\end{eqnarray}

To calculate $\vec\epsilon$, then, we take in a vector of proposed mixing parameters
from MINUIT, calculate the resulting observable parameters from the equations above, 
and subtract the actually observed numbers.

In addition to the fully-general fit allowing all these variables to float, 
there are some variants imposing different no-CPV constraints:
\begin{itemize}
\item No CP violation. In this fit we set $|q/p|=1$, $\phi=0$, and $R_D^+=R_D^-$, 
and fit only for $x$, $y$, $\delta$, and $R_D$. 
\item No direct CP violation. With no direct CP violation, $R_D^+=R_D^-$;
in addition, the four parameters $x$, $y$, $\phi$ and $|q/p|$ are related 
(in the limit that CPV is small) by the constraint
\begin{eqnarray}
|q/p| &=& 1 - \frac{y}{x}\tan\phi \\
\phi &=& \frac{x}{y}\atan\left(\frac{1-|q/p|^2}{1+|q/p|^2}\right).
\end{eqnarray}
Thus we have two variants on this fit:
\begin{description}
\item[2a] Here we allow $|q/p|$ to float and calculate $\phi$
from the constraint. 
\item[2b] We allow $\phi$ to float and calculate $|q/p|$ from the constraint. 
\end{description}
\item All CPV allowed. As $A_D$ is quite
small, the contribution of a new physics phase to $\phi$ is far below
our current sensitivity; consequently the constraint above is a reasonable
approximation. We therefore run three variants of the all-CPV-allowed 
scenario:
\begin{description}
\item[3a] All parameters float, no constraint.
\item[3b] $\phi$ is calculated from $|q/p|$ as above, rather than allowed to float.
$R_D^+$ and $R_D^-$ are both free, as before.
\item[3c] As in 3b, but with $|q/p|$ calculated from the constraint and $\phi$ allowed
to float.
\end{description}
\end{itemize}

In addition, we do a fit not allowing direct CP violation, in which the
free parameters are $x_{12}$, $y_{12}$, and $\phi

\section{Measurements used}



\section{Results}

\section{Conclusion}



\label{lastpage}
\end{document}

\section{Chi-square calculation}
\label{sec:Chi_square_calc}
The purpose of our fit is to combine the errors on several different
measurements of the same parameters, where each measurement may have a
different relation to the underlying true mixing parameters (eg measuring
$(x'^2, y')$ in place of $(x, y)$), and where the numbers in each measurement
may be strongly correlated. To do so we construct an overall $\chi^2$ for
all the results:
\begin{eqnarray}
\chi^2 &=& \vec{\epsilon}^T \sigma^{-1} \vec{\epsilon}
\end{eqnarray}
where the elements of $\vec{\epsilon}$ are given by
$\epsilon_i = m_i - p_i$. Here $\vec{m}$ is the list of measured
values from experiments, and $\vec{p}$ is a set of ``proposed'' values
for the mixing parameters; we use MINUIT to vary $\vec{p}$ so as
to minimise $\chi^2$. Finally, $\sigma$ is an $N\times N$ matrix where
$N$ is the number of measurements, with $\sigma_{ij} = e_i c_{ij} e_j$.
Here $e_i$ is the reported error on measurement $i$, and $c_{ij}$ is the
correlation coefficient between measurements $i$ and $j$. 

Notice that, if
the measurements are uncorrelated, then $\sigma$ reduces to
a diagonal matrix where the elements are the squares of the measurement
errors. In this case $\chi^2$ is simply the sum $\sum\limits_{i}\epsilon_i^2/e_i^2$,
that is, each element is the difference between a measurement
and the corresponding prediction, divided by the error on the measurement,
squared. In other words, if there are no correlations we recover
the usual chi-square goodness-of-fit metric. 

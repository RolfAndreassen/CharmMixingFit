\section{Fit variants}
\label{sec:fit_variants}
In full generality, we wish to fit for no less than seven underlying
related mixing parameters:
\begin{itemize}
\item $x$ and $y$, the normalised mass and width differences
\item $R_D^+$ and $R_D^-$, the ratios of rates
\item $\delta$, the strong phase difference in the $D^0\to K\pi$ channel
\item $|q/p|$ and $\phi$, the magnitude and phase of the indirect CP violation. 
\end{itemize}
The observed inputs, however, are not all direct measurements of these
quantities. From $D^0\to K_S\pi\pi$ we get direct measurements of $x$, $y$, $|q/p|$ and $\phi$;
$D^0\to K\pi$ results also yield $R_D^\pm$ directly, although sometimes quoted
as $R_D = \frac{1}{2}(R_D^+ + R_D^-)$ and $A_D = \frac{R_D^+ - R_D^-}{R_D^+ + R_D^-}$.
However, we also measure the derived parameters $x'^{2(\pm)}$, $y'^{(\pm)}$, $y_{CP}$, and $A_\Gamma$,
defined as:
\begin{eqnarray}
x' &=& x\cos\delta + y\sin\delta \\
y' &=& y\cos\delta - x\sin\delta \\
x'^{\pm} &=& \left(\frac{1\pm A_M}{1\mp A_M}\right)^{1/4}\left(x'\cos\phi \pm y'\sin\phi\right) \\
y'^{\pm} &=& \left(\frac{1\pm A_M}{1\mp A_M}\right)^{1/4}\left(y'\cos\phi \mp x'\sin\phi\right) \\
2y_{CP}  &=& \left(|q/p|+|p/q|\right)y\cos\phi - \left(|q/p|-|p/q|\right)x\sin\phi\\
2A_\Gamma  &=& \left(|q/p|-|p/q|\right)y\cos\phi - \left(|q/p|+|p/q|\right)x\sin\phi\\
\end{eqnarray}
where the helper quantity $A_M$ is given by
\begin{eqnarray}
A_M &=& \frac{|q/p|^2 - |p/q|^2}{|q/p|^2 + |p/q|^2}. 
\end{eqnarray}

To calculate $\vec\epsilon$, then, we take in a vector of proposed mixing parameters
from MINUIT, calculate the resulting observable parameters from the equations above, 
and subtract the actually observed numbers.

In addition to the fully-general fit allowing all these variables to float, 
there are some variants imposing different no-CPV constraints:
\begin{itemize}
\item No CP violation. In this fit we set $|q/p|=1$, $\phi=0$, and $R_D^+=R_D^-$, 
and fit only for $x$, $y$, $\delta$, and $R_D$. 
\item No direct CP violation. With no direct CP violation, $R_D^+=R_D^-$;
in addition, the four parameters $x$, $y$, $\phi$ and $|q/p|$ are related 
(in the limit that CPV is small) by the constraint
\begin{eqnarray}
%|q/p| &=& 1 - \frac{y}{x}\tan\phi\\
|q/p| &=& \sqrt{\frac{x - y\tan\phi}{x + y\tan\phi}}\\
\phi &=& \tan^{-1}\left(\frac{1-|q/p|^2}{1+|q/p|^2}\frac{x}{y}\right)
\end{eqnarray}
Thus, to extract errors on both variables, we have two variants on this fit:
\begin{description}
\item[2a] Here we allow $|q/p|$ to float and calculate $\phi$
from the constraint. 
\item[2b] We allow $\phi$ to float and calculate $|q/p|$ from the constraint. 
\end{description}
\item All CPV allowed. As $A_D$ is quite
small, the contribution of a new physics phase to $\phi$ is far below
our current sensitivity; consequently the constraint above is a reasonable
approximation. We therefore run three variants of the all-CPV-allowed 
scenario:
\begin{description}
\item[3a] All parameters float, no constraint.
\item[3b] $\phi$ is calculated from $|q/p|$ as above, rather than allowed to float.
$R_D^+$ and $R_D^-$ are both free, as before.
\item[3c] As in 3b, but with $|q/p|$ calculated from the constraint and $\phi$ allowed
to float.
\end{description}
\end{itemize}

In addition, we do a fit not allowing direct CP violation, in which the
free parameters are the underlying\footnote{See Kagan and Sokoloff, Phys.Rev.D80:076008 (2009), \url{http://arxiv.org/abs/0907.3917}.}
$x_{12}$, $y_{12}$, and $\phi_{12}$. These parameters are related
(in the limit of no direct CP violation) to $|q/p|$, $x$, $y$ and $\phi$ (no subscripts!) as follows:
\begin{eqnarray}
x &=& \frac{1}{\sqrt{2}}\mathrm{sign}(\cos\phi_{12})
\sqrt{x_{12}^2 - y_{12}^2 + |x_{12}^2+y_{12}^2| - 4x_{12}^2y_{12}^2\sin^2(\phi_{12})} \\
%
y &=& \frac{1}{\sqrt{2}}
\left(y_{12}^2 - x_{12}^2 + \sqrt{(x_{12}^2+y_{12}^2)^2 - 4x_{12}^2y_{12}^2\sin^2(\phi_{12})}\right)^{1/2} \\
%
|q/p| &=& \left(\frac{x_{12}^2+y_{12}^2+2x_{12}y_{12}\sin(\phi_{12})}
{x_{12}^2+y_{12}^2-2x_{12}y_{12}\sin(\phi_{12})}\right)^{1/4}\\
%
\phi &=& -\frac{1}{2}\frac{\sin(2\phi_{12})}{\cos(2\phi_{12})+\frac{y_{12}^2}{x_{12}^2}}.
\end{eqnarray}
Our approach in this fit is to allow MINUIT to believe that the parameters $x_{12}$, $y_{12}$,
and $\phi_{12}$ are free, but calculate the
non-underlying\footnote{Overlying?} parameters $x$, $y$, $\phi$, and $|q/p|$, and use these
values in the calculation of $\chi^2$, as outlined for the other fits. 


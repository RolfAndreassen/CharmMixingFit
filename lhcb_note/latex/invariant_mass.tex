\section{Invariant Mass Plots}
\label{sec:Invariant-Mass-Plots}
\hspace{3mm} As a first check of the Data-Driven study, a check of the 
$\mu\mu_\text{Extra}$ invariant mass is performed. From earlier studies (cite 
SemiLeptonic $\Delta\mathcal{A}_{CP}$ paper), it is understood that the contribution 
of $J/\psi$ to muon misidentification is at the per-mille level. However, 
instead of swapping muons, the study focused on misidentification of muons as kaon 
candidtates. In order to expand the study, the invariant mass of the original and extra 
muons is plotted in Figure~\ref{fig:mumumass}. Separating further by charge of the 
extra muon, a peak at 3.1 GeV is seen, constituting about 0.5\% of the event sample. \\
%Hence, this contribution is roughly at the half percent level, which is in agreement 
%with the previous studies.\\
\hspace{3mm}Additionally, it is possible to swap the extra muon with the original muon 
and explore the invariant mass of the $\mu_\text{Extra} D^*$ system. This is plotted in 
Figure~\ref{fig:dstar-muext}. Here, an unusual double-peaking structure is seen at 
invariant masses lower than 3.1 GeV. This is not identifiable with any resonance, and 
should not be peaking if the candidate is actually a muon. By replacing the hypothesis 
of the particle with a kaon mass hypothesis, a single peaking structure is then seen in 
Figure~\ref{fig:dstar-muext-k}. The resulting peak is well below the minimum analysis 
cut. However, the combinatoric exponential tail on the high mass side is present for 
both situations and is constant over the entire analysis range for both charges of muon 
considered. This represents the random muon background.\\
As an additonal check, the $\mu_\text{Extra} D^*$ is checked assuming a proton mass
hypothesis in place of the muon mass hypothesis. The results are shown in
Figure~\ref{fig:dstar-muext-proton}. No peaking structure is seen.
%%%%%%mu mu%%%%%%%%%
\begin{figure}[tb]
  \begin{center}
	\includegraphics[width=0.49\linewidth]{rs_loop_dimu_mass_extramu_nocut} \put(-58,123){(a)}
	\includegraphics[width=0.49\linewidth]{rs_loop_dimu_mass_extramu_nocut_logy} \put(-58,123){(b)}
	\end{center}
  \caption{
    \small %captions should be a little bit smaller than main text                                                                                                                
    Invariant mass distributions of $m(\mu\mu_\text{Extra})$ in (a) linear scale and (b) log scale. The peak at 3.1 GeV/c$^2$ is the $J/\psi$ resonance. 
    }
  \label{fig:mumumass}
\end{figure}

%%%%%%%%%%mu dstar %%%%%%%%%%%

\begin{figure}[tb]
  \begin{center}
	\includegraphics[width=0.49\linewidth]{rs_loop_bmass_extramu_nocut} \put(-58,123){(a)}
	\includegraphics[width=0.49\linewidth]{rs_loop_bmass_extramu_nocut_logy} \put(-58,123){(b)}
	\end{center}
  \caption{
    \small %captions should be a little bit smaller than main text                                                                                                                
    Invariant mass distributions of $m(D^{*}\mu_\text{Extra})$ in (a) linear scale and (b) log scale. Peaking structures which are unassociated with clear resonances are seen.
    }
  \label{fig:dstar-muext}
\end{figure}
%%%%%%%%%%mu dstar , K hypothesis%%%%%%%%%%%

\begin{figure}[tb]
  \begin{center}
	\includegraphics[width=0.49\linewidth]{rs_loop_bmass_extramu_khypo} \put(-58,103){(a)}
	\includegraphics[width=0.49\linewidth]{rs_loop_bmass_extramu_khypo_logy} \put(-58,103){(b)}
	\end{center}
  \caption{
    \small %captions should be a little bit smaller than main text                                                                                                                
    Invariant mass distributions of $m(D^{*}\mu_\text{Extra})$ with a kaon mass hypothesis for the muon. Shown in (a) linear scale and (b) log scale. By swapping the mass hypothesis, it is possible to identify the resonance as a XXXX.
    }
  \label{fig:dstar-muext-k}
\end{figure}

%%%%%%%%%%mu dstar , proton hypothesis%%%%%%%%%%%

\begin{figure}[tb]
  \begin{center}
	\includegraphics[width=0.49\linewidth]{rs_loop_bmass_extramu_proton_hypo} \put(-58,103){(a)}
	\includegraphics[width=0.49\linewidth]{rs_loop_bmass_extramu_proton_hypo_logy} \put(-58,103){(b)}
	\end{center}
  \caption{
    \small %captions should be a little bit smaller than main text                                                                                                                
    Invariant mass distributions of $m(D^{*}\mu_\text{Extra})$ with a proton mass hypothesis for the muon. Shown in (a) linear scale and (b) log scale. No resonant structure is seen.
    }
  \label{fig:dstar-muext-proton}
\end{figure}


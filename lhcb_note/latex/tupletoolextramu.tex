\section{TupleToolExtraMu}
\label{sec:TupleToolExtraMu}
Looking explicitly for extra particles in the decay and asking whether or not these 
particles would satisfy the selection criteria for a given channel is done automatically 
by the DecayTreeTuple algorithm. However, it is beneficial to ask whether there are other 
particles which would satisfy the same selection criteria. The motivation behind writing 
the tool TupleToolExtraMu is to explicitly search for extra muons in an event which could 
be confused with, or be mis-associated as the true muon from the semileptonic decay 
$B\to \mu D^* X$. Storing additonal information about the ``extra" candidate then allows 
for a by-hand analysis of extra candidates.\\
TupleToolExtraMu configures as a tuple tool added to the DecayTreeTuple algorithm. By default, 
it outputs
\begin{itemize}
	\item The number of additional muons reconstructed from the TES location StdAllVeryLooseMuons, called extra muons
	\item The 4-momentum components of the extra muons
	\item The charge of the extra muon
	\item The Delta Log Likelihood distributions $\Delta \log( \mathcal{L}(\mu-\pi))$ and  $\Delta \log( \mathcal{L}(\mu-K))$ for the extra muon
	\item The Ghost Probability of the extra muon
	\item The extra muon Track Ghost Probability
	\item Extra muon Track $\chi^2/DoF$
	\item The invariant mass of the extra muon and the original muon being used
	\item The vertex $\chi^2$ and $DoF$ of the extra muon and original muon vertex
	\item The vertex $\chi^2$ and $DoF$ of the extra muon and the $D^*$ candidate
	\item The invariant mass of the extra muon and $D^*$ candidates
	\item The IP and IP$\chi^2$ of the extra muon
	\item Whether or not the extra muon originates from the same primary vertex as the $D^*$ candidate
\end{itemize}
This information allows one to, by hand, apply the stripping selections to the extra muons. 
The inclusion of the invariant mass distributions is meant to provide a tool in searching for 
additional sources of background. Figure~\ref{fig:nextra} shows the output of the number of 
extra muons summed over all events. The additional lines show the transformation of the number 
of additional candidate muons after applying cuts.

%%%%%%nextramu distributions%%%%%%%%%
\begin{figure}[tb]
  \begin{center}
	\includegraphics[width=0.49\linewidth]{rs_n_extra_muon_stripped} \put(-58,123){(a)}
	\includegraphics[width=0.49\linewidth]{rs_n_extra_muon_stripped_logy} \put(-58,123){(b)}
	\end{center}
  \caption{
    \small %captions should be a little bit smaller than main text                                                                                                                
    Distribution of $n(\mu_\text{Extra})$, the number of extra muons in a event, in (a) linear scale and (b) log scale. Different selection cuts are applied to extra muons to determine what fraction mis-association is to be expected.
    }
  \label{fig:nextra}
\end{figure}
